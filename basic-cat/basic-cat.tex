%%% Local Variables:
%%% coding: utf-8
%%% End:
\documentclass[11pt]{article}

\usepackage{amsmath}

\usepackage{fancyhdr}
\usepackage{amssymb}
\usepackage{amsthm}

\title{My Exercise Answers to
Basic Category Theory }
\author{Javran Cheng}

\begin{document}

\maketitle

\section{Category and Functors}

\subsection{Exercises}

\paragraph{Exercise 1}

Assume there are more than one arrow whose domain and codomain are all $X$,
we are going to prove by contradiction that the arrow $id_X$ is unique.

Pick up two such arrows: $id1_X$ and $id2_X$ and $id1_x \neq id2_x$.

According to the property of identity arrow:

\begin{align*}
id1_X \circ id2_X & = id1_X \\
id1_X \circ id2_X & = id2_X \\
\therefore id1_X & = id2_X
\end{align*}

This conflicts with our assumption that $id1_x \neq id2_x$,
thus the identity function $id_X$ is unique.

\paragraph{Exercise 2}

\subparagraph{Proof a preorder forms a category}

We check every property of category holds.

\begin{itemize}
\item{object}: $\forall x, x \in X$ are the objects
\item{morphism}: Given any relation $x \le y$,
there is an morphism from $x$ to $y$.
\item{composition}: If $f: x \rightarrow y$ and $g: y \rightarrow z$,
the compostion is $g \circ f: x \rightarrow z$. And $g \circ f$ corresponses to
$x \le z$, which is true by the transitivity of the preorder set.
item{associativity}: Suppose we have:
\begin{align*}
f &: x \rightarrow y & (x \le y) \\
g &: y \rightarrow z & (y \le z) \\
h &: z \rightarrow w & (z \le w)
\end{align*}
Therefore:
\begin{align*}
g \circ f &: x \rightarrow z & (x \le z) \\
h \circ g &: y \rightarrow w & (y \le w) \\
h \circ (g \circ f) &: x \rightarrow w & (x \le w) \\
(h \circ g) \circ f) &: x \rightarrow w & (x \le w)
\end{align*}

The associativity of composition holds because
$h \circ (g \circ f) = (h \circ g) \circ f$
\item{identity}: $\forall x \in X, id_x: x \rightarrow x \quad (x \le x)$
\end{itemize}

\subparagraph{Proof the converse}

To prove the converse, we need to check every property of preorder set holds for
$\mathcal{C}_0$, if $\mathcal{C}$ is a category.

\begin{itemize}
\item{element}: the set elements are all objects in $\mathcal{C}$.
\item{binary relation}: each morphism $f: a \rightarrow b$ in $\mathcal{C}$
corresponses to a pair $(a,b) \in (\le)$, or $a \le b$.
\item{reflexivity}: for each object $x \in \mathcal{C}_0$, there is an identity morphism
$id_x$. This morphism corresponses to the relation $x \le x$.
\item{transitivity}: $\forall a,b,c,f,g \in \mathcal{C}$, if
 $f: a \rightarrow b$ and $g: b \rightarrow c$, then $g \circ f: a \rightarrow c$.
This means in $\mathcal{C}_0$, $\forall a,b,c \in \mathcal{C}_0$,
if $a \le b$ and $b \le c$, then $a \le c$. Thus transitivity holds for $\le$.
\end{itemize}

\paragraph{Exercise 3}

Despite the exercise does not define explicitly
about what is \textit{substring}, I guess a \textit{substring}
should be consecutive.
For example, ``xy'' is a substring of ``xxyy'', but is not a substring of ``xzzy''.

\subparagraph{Proof $\langle \tilde{A}, \star \rangle$ is a group}

Let $\mathop{concat}(X_1,X_2,\ldots,X_n)$ be the concatenation of strings
$X_1,X_2\ldots,X_n$ in order.

Let $\mathop{rm}(X)$ be the result of the operation of successively removing form $xx^{-1}$
and $x^{-1}x$ from $X$ until $X \in \tilde{A}$.

Therefore $X \star Y = \mathop{rm}(\mathop{concat}(X,Y))$.

Let the corresponding element of a character $c$ be $\mathop{inv}(c)$.
i.e. $inv(c) = c^{-1}, inv(c^{-1}) = c$.

\begin{proof}
For any given string $X = a_1,a_2,\ldots,a_n$ on alphabet $A \cup A^{-1}$,
$rm(X)$ can only have one result. In other words, $rm$ is a function.

First observe that for any given $X$, we can perform finite steps
of deletion to get a result $X'$ in $\tilde{A}$. Then we can do the
reverse: starting from $X'$, we take steps by putting substring of form $x x^{-1}$ or
$x^{-1} x$ into it. After a finite steps, $X$ should be contained in the list of
possible results because it only takes a finite step for $X$ to reach $X'$,
so does the reverse.

Through these operations, in any given step, let $X_1$ be the current
string before insertion, and $X_2$ be the result after insertion,
we will show that no matter how the insertion is performed, operation $rm(X_2)$
will have only one result. Therefore for any given $X$, $rm(X)$ can only have
one result, and therefore $rm(X)$ is a function of $X$.

For every step, we insert a single pair, either of form $x x^{-1}$ or of form
$x^{-1} x$.

We do case analysis on the insertion step:

\paragraph{Case 1: the pair are inserted before or after the whole string}
Consider $c \mathop{inv}(c)$ is inserted before the whole string $a_1,a_2,\ldots,a_n$,
if the string is empty or if $a_1 \neq c$, there is no way to divert the result.
If $a_1 = c$, then the resulting string is $c,inv(c),c,a_2,\ldots,a_n$.
Although there are two ways of deleting character pair on this part of string,
the result will always be $c,a_2,\ldots,a_n$. Because the operation is symmetric,
the same property holds if we are inserting pairs after the whole string.

\paragraph{Case 2: the pair are inserted between two characters}
Consider $c \mathop{inv}(c)$ is inserted between $a_i$ and $a_j$ in string
$a_1,\ldots,a_i,a_j,\ldots,a_n$.

\begin{itemize}
  \item{$a_i = inv(c), a_j = c$}

    The result will be $a_1,\ldots,inv(c),c,inv(c),c,\ldots,a_n$,
    they are several ways of removing pairs from it, but all the deletions
    will end up with the same result. (There are certainly some cases concerning
    the $\ldots$ parts in $a_1,\ldots,a_i$, but I do not think this will make any difference
    about the conclusion).

  \item{$a_i = inv(c), a_j \neq c$}

    No matter how the delection is performed, the $a_i\ldots a_j$ portion will
    always be $inv(c),a_j$ after delection.

  \item{$a_i \neq inv(c), a_j = c$ or $a_i \neq inv(c), a_j \neq c$}

    Similiar result holds.
\end{itemize}

Therefore for any $X$, $rm(X)$ is a function.
\end{proof}

In addition, regarding the property of this delection. We know that:
 $\mathop{rm}(\mathop{concat}(\mathop{rm}(X),\mathop{rm}(Y)))
=\mathop{rm}(\mathop{concat}(X,Y))$.

Now proof $\langle \tilde{A}, \star \rangle$ is a group by the definition of group.

\begin{itemize}
  \item{Closure}: the definition of $\star$ shows the result must be
    an element of $\tilde{A}$.
  \item{Associativity}:
    \begin{align*}
      (X \star Y) \star Z
      & = \mathop{rm}(\mathop{concat}(\mathop{rm}({\mathop{concat}}(X,Y)),Z)) \\
      & = \mathop{rm}(\mathop{concat}(\mathop{rm}({\mathop{concat}}(X,Y)),\mathop{rm}(Z))) \\
      & = \mathop{rm}(\mathop{rm}(\mathop{concat}(\mathop{concat}(X,Y),Z))) \\
      & = \mathop{rm}(\mathop{concat}(X,Y,Z)) \\
      X \star (Y \star Z)
      & = \mathop{rm}(\mathop{concat}(X,\mathop{rm}({\mathop{concat}}(Y,Z)))) \\
      & = \mathop{rm}(\mathop{concat}(\mathop{rm}(X),\mathop{rm}({\mathop{concat}}(Y,Z)))) \\
      & = \mathop{rm}(\mathop{rm}(\mathop{concat}(X,\mathop{concat}(Y,Z)))) \\
      & = \mathop{rm}(\mathop{concat}(X,Y,Z))
    \end{align*}
    Therefore $(X \star Y) \star Z = X \star (Y \star Z)$.
  \item{Identity element}:
    The empty string $\varnothing$ is the identity element.
  \item{Inverse element}:
    The inverse element for empty string is the empty string.
    For any other given string $a_1,a_2,\ldots,a_n$ in $\tilde{A}$,
    its inverse is $\mathop{inv}(a_n),\mathop{inv}(a_{n-1}),\ldots,\mathop{inv}(a_1)$.
    Because $a_i\mathop{inv}(a_i)$ and $\mathop{inv}(a_i)a_i$ are of the form
    $xx^{-1}$ and $x^{-1}x$. We know for any string $X$ and its inverse $Y$ in $\tilde{A}$,
    $X \star Y = \varnothing = Y \star X$.
\end{itemize}

\subparagraph{Prove the assignment $A \mapsto \tilde{A}$ is part of a functor:
$Set \rightarrow Grp$}

Consider these factors:

\begin{itemize}
  \item{$A$ is surely a set}
  \item{$\tilde{A}$ forms a group: $\langle \tilde{A}, \star \rangle$}
\end{itemize}

By the definition of functor, we know the assignment is part of a functor
from $Set$ to $Grp$.

\subsection{Some notes}

\paragraph{Composition implies existency}
For two morphisms $f$ and $g$ of a category $\mathcal{C}$, the result of composing $g \circ f$
must also be a morphism of $\mathcal{C}$.

\end{document}