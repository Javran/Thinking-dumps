%%% Local Variables:
%%% coding: utf-8
%%% End:
\documentclass[11pt]{article}

\usepackage{amsmath}

\usepackage{fancyhdr}
\usepackage{amssymb}
\usepackage{amsthm}

\title{My Exercise Answers to
Basic Category Theory }
\author{Javran Cheng}

\begin{document}

\maketitle

\section{Category and Functors}

\subsection{Exercises}

\paragraph{Exercise 1}

Assume there are more than one arrow whose domain and codomain are all $X$,
we are going to prove by contradiction that the arrow $id_X$ is unique.

Pick up two such arrows: $id1_X$ and $id2_X$ and $id1_x \neq id2_x$.

According to the property of identity arrow:

\begin{align*}
id1_X \circ id2_X & = id1_X \\
id1_X \circ id2_X & = id2_X \\
\therefore id1_X & = id2_X
\end{align*}

This conflicts with our assumption that $id1_x \neq id2_x$,
thus the identity function $id_X$ is unique.

\paragraph{Exercise 2}

\subparagraph{Proof a preorder forms a category}

We check every property of category holds.

\begin{itemize}
\item{object}: $\forall x, x \in X$ are the objects
\item{morphism}: Given any relation $x \le y$,
there is an morphism from $x$ to $y$.
\item{composition}: If $f: x \rightarrow y$ and $g: y \rightarrow z$,
the compostion is $g \circ f: x \rightarrow z$. And $g \circ f$ corresponses to
$x \le z$, which is true by the transitivity of the preorder set.
item{associativity}: Suppose we have:
\begin{align*}
f &: x \rightarrow y & (x \le y) \\
g &: y \rightarrow z & (y \le z) \\
h &: z \rightarrow w & (z \le w)
\end{align*}
Therefore:
\begin{align*}
g \circ f &: x \rightarrow z & (x \le z) \\
h \circ g &: y \rightarrow w & (y \le w) \\
h \circ (g \circ f) &: x \rightarrow w & (x \le w) \\
(h \circ g) \circ f) &: x \rightarrow w & (x \le w)
\end{align*}

The associativity of composition holds because
$h \circ (g \circ f) = (h \circ g) \circ f$
\item{identity}: $\forall x \in X, id_x: x \rightarrow x \quad (x \le x)$
\end{itemize}

\subparagraph{Proof the converse}

To prove the converse, we need to check every property of preorder set holds for
$\mathcal{C}_0$, if $\mathcal{C}$ is a category.

\begin{itemize}
\item{element}: the set elements are all objects in $\mathcal{C}$.
\item{binary relation}: each morphism $f: a \rightarrow b$ in $\mathcal{C}$
corresponses to a pair $(a,b) \in (\le)$, or $a \le b$.
\item{reflexivity}: for each object $x \in \mathcal{C}_0$, there is an identity morphism
$id_x$. This morphism corresponses to the relation $x \le x$.
\item{transitivity}: $\forall a,b,c,f,g \in \mathcal{C}$, if
 $f: a \rightarrow b$ and $g: b \rightarrow c$, then $g \circ f: a \rightarrow c$.
This means in $\mathcal{C}_0$, $\forall a,b,c \in \mathcal{C}_0$,
if $a \le b$ and $b \le c$, then $a \le c$. Thus transitivity holds for $\le$.
\end{itemize}

\newpage
\paragraph{Exercise 3}

Despite the exercise does not define explicitly
about what is \textit{substring}, I guess a \textit{substring}
should be consecutive.
For example, ``ab'' is a substring of ``aabb'', but is not a substring of ``accb''.

Let $\mathop{concat}(A_1,A_2,\ldots,A_n)$ be the concatenation of strings
$A_1,A_2\ldots,A_n$ in order.

Let $\mathop{rm}(A)$ be the result of the operation of successively removing form $xx^{-1}$
and $x^{-1}x$ from A until $A \in \tilde{A}$.

Let the corresponding element of a character $a$ be $\mathop{inv}(a)$.
i.e. $inv(a) = a^{-1}, inv(a^{-1}) = a$.

Therefore $A \star B = \mathop{rm}(\mathop{concat}(A,B))$.

For now, I just assume for all possible $A$, $\mathop{rm}(A)$ is unique, i.e. $\mathop{rm}$
is a function. (Since I don't know how to prove that.)

Another assumption: $\mathop{rm}(\mathop{concat}(\mathop{rm}(A),\mathop{rm}(B)))
=\mathop{rm}(\mathop{concat}(A,B))$.

% TODO: 'A' is also used for the set name..

\subparagraph{Proof $\langle \tilde{A}, \star \rangle$ is a group}

Prove by the definition of group.

\begin{itemize}
  \item{Closure}: the definition of $\star$ shows the result must be
    an element of $\tilde{A}$.
  \item{Associativity}:
    \begin{align*}
      (A \star B) \star C
      & = \mathop{rm}(\mathop{concat}(\mathop{rm}({\mathop{concat}}(A,B)),C)) \\
      & = \mathop{rm}(\mathop{concat}(\mathop{rm}({\mathop{concat}}(A,B)),\mathop{rm}(C))) \\
      & = \mathop{rm}(\mathop{rm}(\mathop{concat}(\mathop{concat}(A,B),C))) \\
      & = \mathop{rm}(\mathop{concat}(A,B,C)) \\
      A \star (B \star C)
      & = \mathop{rm}(\mathop{concat}(A,\mathop{rm}({\mathop{concat}}(B,C)))) \\
      & = \mathop{rm}(\mathop{concat}(\mathop{rm}(A),\mathop{rm}({\mathop{concat}}(B,C)))) \\
      & = \mathop{rm}(\mathop{rm}(\mathop{concat}(A,\mathop{concat}(B,C)))) \\
      & = \mathop{rm}(\mathop{concat}(A,B,C))
    \end{align*}
    Therefore $(A \star B) \star C = A \star (B \star C)$.
  \item{Identity element}:
    The empty string $\varnothing$ is the identity element.
  \item{Inverse element}:
    The inverse element for empty string is the empty string.
    For any other given string $a_1,a_2,\ldots,a_n$ in $\tilde{A}$,
    its inverse is $\mathop{inv}(a_n),\mathop{inv}(a_{n-1}),\ldots,\mathop{inv}(a_1)$.
    Because $a_i\mathop{inv}(a_i)$ and $\mathop{inv}(a_i)a_i$ are of the form
    $xx^{-1}$ and $x^{-1}x$. We know for any string $A$ and its inverse $B$ in $\tilde{A}$,
    $A \star B = \varnothing = B \star A$.
\end{itemize}

\subsection{Some notes}

\paragraph{Composition implies existency}
For two morphisms $f$ and $g$ of a category $\mathcal{C}$, the result of composing $g \circ f$
must also be a morphism of $\mathcal{C}$.

\end{document}