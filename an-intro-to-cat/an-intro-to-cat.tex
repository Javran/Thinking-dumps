\documentclass[11pt]{article}

\usepackage{amsmath}

\usepackage{fancyhdr}
\usepackage{amssymb}
\usepackage{amsthm}

\usepackage{diagrams}

\title{My Exercise Answers to \\
Introduction to Category Theory}
\author{Javran Cheng}

\begin{document}

\maketitle

\newcommand{\cat}[1]{\textit{#1}}
\newcommand{\fcomp}{\circ}

\newcommand{\nat}{\mathbb{N}}

\section{Categories}

\subsection{Categories defined}

\subsubsection{Observe that sets and functions do form a category \cat{Set}}

\begin{itemize}
  \item Objects are sets.
  \item Arrows are total functions, because function are just binary relations
    between two sets.
  \item Composition is the function composition.
  \item Associativity holds because $(h \fcomp g) \fcomp f = h \fcomp (g \fcomp f)$
    always holds for compatible functions.
  \item Identity arrows are functions that relates each element of a set to itself.
    Therefore the requirement $id_B \fcomp f = f = f \fcomp id_A$ becomes trivial.
\end{itemize}

\subsubsection{Each poset is a category, and each monoid is a category}

\paragraph{Each poset is a category} For any given poset, each element in it
is the corresponding object in the category. And each $\le$ relation corresponses
to an arrow in the category. Since $\le$ relation is transitive, the associativity
of arrow composition holds. And the identity for each object $A$ in the category is $(A,A)$,
this is because $A \le A$ holds (The binary relation is reflexive).

\paragraph{Each monoid is a category} For any given monoid $(M, \star , 1)$, the corresponding
category has only one element, which is the monoid itself. Every element in
this monoid is an arrow from the only object to itself. The arrow composition is
the binary relation $\star$. (We can observe that for any two arrows $f$ and $g$, $f \star g$
does form another arrow, and that $(f \star g) \star h = f \star (g \star h)$
is guaranteed by the property held by the monoid).

\subsection{Categories of structured sets}

\subsubsection{The category \cat{Pno}}

\paragraph{(a) Verify that \cat{Pno} is a category}

\begin{itemize}
  \item Objects are $(A, \alpha, a)$ described in the question.
  \item Arrows are functions $f : A \rTo B$ as described in the question
  \item Arrow composition.
    Suppose two arrows are:
    \begin{diagram}
      (A,\alpha,a) & & \rTo^f & &
      (B,\beta,b)  & & \rTo^g & &
      (C,\gamma,c)
    \end{diagram}
    Where $f$ and $g$ preserves the structure:
    \begin{align*}
      f \fcomp \alpha & = \beta  \fcomp f & \quad f(a) = b \\
      g \fcomp \beta  & = \gamma \fcomp g & \quad g(b) = c
    \end{align*}
    We can observe that:
    \begin{align*}
        & g \fcomp f \fcomp \alpha \\
      = & g \fcomp \beta  \fcomp f \\
      = & \gamma \fcomp g \fcomp f
    \end{align*}
    and that:
    \begin{align*}
      (g \fcomp f)(a) = g(b) = c
    \end{align*}

    Therefore the arrow composition is the function composition.

  \item Associativity holds by the function composition.
  \item For each object $(A,\alpha, a)$, $id_A \fcomp \alpha = \alpha \fcomp id_A$ and
    $id_A(a) = a$. So the identity arrow is the identity function on $A$ viewed as a morphism.
\end{itemize}

Therefore $\cat{Pno}$ is a category.

\paragraph{(b) Show that $(\nat, succ, 0)$ is a $\cat{Pno}$-object}

\begin{itemize}
  \item $\nat$ is a set.
  \item $succ$ is of type $\nat \rTo \nat$.
  \item $0 \in \nat$ is a nominated element.
\end{itemize}

Therefore, $(\nat, succ, 0)$ is a $\cat{Pno}$-object.

\paragraph{(c) Unique arrow and the behavior of the carrying function}\mbox{}

We know that $succ(0) = 1, succ(1) = 2, \ldots, succ(n) = n + 1$.
In order to describe the behavior of the carrying function $\alpha$,
We observe the value of $\alpha(f(0)), \alpha(f(1)), \ldots, \alpha(f(n))$:

\begin{align*}
\alpha(f(0)) & = (f \fcomp succ)(0) = (f \fcomp succ^1)(0) \\
\alpha(f(1)) & = \alpha(f \fcomp succ (0)) = (f \fcomp succ \fcomp succ)(0)
= (f \fcomp succ^2)(0) \\
& \ldots
\end{align*}

To sum up, $\forall n \in \nat$:
\begin{align*}
& \alpha(f(n)) \\
= &(\alpha \fcomp f)(succ^n(0)) \\
= &(\alpha \fcomp f \fcomp succ^n) (0) \\
= &(f \fcomp succ \fcomp succ^n) (0) \\
= &(f \fcomp succ^{n+1})(0) \\
= &f(n+1)
\end{align*}

And also:

\begin{align*}
& f(n) \\
= & (f \fcomp succ^{n})(0) \\
= & (\alpha^n \fcomp f)(0) \\
= & \alpha^n(a)
\end{align*}

Therefore the behavior of $\alpha$ is: $\forall a_1 \in A, \exists n_1 \in \nat$
s.t. $f(n_1) = a_1$, applying $\alpha$ to $a_1$ produces the same result as
applying $succ$ to $n_1$ and then applying $f$ to the resulting value does.

We can informally says that $\alpha$ to $(A,a)$ is like $succ$ to $(\nat,0)$.

The uniqueness of $f$ is also established, because from $f(n) = \alpha^n(a)$,
we know that $f$ is totally determined by $\alpha$ and $a$.

\subsubsection{Category $\cat{SetD}$}

\begin{itemize}
  \item Objects are of form $(A,X)$ which $A$ is a set and $X \subseteq A$.
  \item Arrows are functions of type $f: A \rTo B$ between two objects
    $(A,X)$ and $(B,Y)$
    where $\forall x \in X, f(x) \in Y$.
  \item Arrow composition. For any two compatible arrows $f$, $g$,
    we have the diagram:

    \begin{diagram}
      (A,X) & & \rTo^f & &
      (B,Y) & & \rTo^g & &
      (C,Z)
    \end{diagram}

    For arrow $f$ and $g$, the following constraints hold, respectively:

    \begin{align*}
      \forall x \in X, f(x) = y_x, y_x \in Y \quad \forall y \in Y, g(y) = z_y, z_y \in Z
    \end{align*}

    Therefore:

    \begin{align*}
      \forall x \in X, g(f(x)) = (g \fcomp f)(x) \in Z
    \end{align*}

    Therefore the arrow composition is function composition.

  \item Composition associativity. The associativity holds because
    the arrow composition is function composition.
  \item Identity arrow is the identity function viewed as a morphism.
\end{itemize}

\end{document}