\documentclass[11pt]{article}

\usepackage{amsmath}

\usepackage{fancyhdr}
\usepackage{amssymb}
\usepackage{amsthm}
\usepackage{xspace}
\usepackage{indentfirst}

\usepackage{diagrams}

\title{My Exercise Answers to \\
Introduction to Category Theory}
\author{Javran Cheng}

\begin{document}

\maketitle

\newcommand{\cat}[1]{\textit{#1}}
\newcommand{\fcomp}{\circ}

\newcommand{\nat}{\mathbb{N}}

\section{Categories}

\subsection{Categories defined}

\subsubsection{Observe that sets and functions do form a category \cat{Set}}

\begin{itemize}
  \item Objects are sets.
  \item Arrows are total functions, because function are just binary relations
    between two sets.
  \item Composition is the function composition.
  \item Associativity holds because $(h \fcomp g) \fcomp f = h \fcomp (g \fcomp f)$
    always holds for compatible functions.
  \item Identity arrows are functions that relates each element of a set to itself.
    Therefore the requirement $id_B \fcomp f = f = f \fcomp id_A$ becomes trivial.
\end{itemize}

\subsubsection{Each poset is a category, and each monoid is a category}

\paragraph{Each poset is a category} For any given poset, each element in it
is the corresponding object in the category. And each $\le$ relation corresponses
to an arrow in the category. Since $\le$ relation is transitive, the associativity
of arrow composition holds. And the identity for each object $A$ in the structured set is $(A,A)$,
this is because $A \le A$ holds (The binary relation is reflexive).

\paragraph{Each monoid is a category} For any given monoid $(M, \star , 1)$, the corresponding
category has only one object (Here we don't need to care about
what exactly the object is, which isn't important). Every element in
this monoid is an arrow from the only object to itself.
The arrow composition is
the binary relation $\star$. We can observe that for any two arrows $f$ and $g$, $f \star g$
does form another arrow, and that $(f \star g) \star h = f \star (g \star h)$
is guaranteed by the property held by the monoid.
And because the unit $1$ correponses to the identity arrow,
the requirement that $1 \star m = m = m \star 1$ holds trivially (where $m$ are
elements in the monoid).
\input{exercise-1-2s.tex}

\subsection{An arrow need not be a function}

\subsubsection{Show that strictly positive integers as objects
and matrices as arrows gives a category}

\begin{itemize}
  \item Objects are strictly positive integers.
  \item Let $m$ and $n$ be two objects, the arrow between them
    ($n \rTo^A m$)
    is an $m \times n$ matrix $A$.
  \item Given two compatible matrices:
    $n \rTo^B k$ and $k \rTo^A m$,
    the arrow composition is given by the matrix product $AB$:
    $n \rTo^{A \fcomp B} m$.
  \item The associativity holds because for any three compatible matrices $A$,$B$ and $C$,
    the matrix multiplication is associative: $(AB)C = A(BC)$.
    Therefore the associativity for arrow composition is obtained from the matrix
    multiplication.
  \item For any object $n$, the identity arrow is the identity matrix $E_n$.
    It holds that $f \fcomp id_n = f = id_m \fcomp f$ for any arrow $f : n \rTo^A m$
    ($A$ is the corresponding matrix of $f$) because of the following property for
    matrix multiplication holds: $ A \times E_n = A = E_m \times A$.
\end{itemize}

Therefore this gives a category. If we have to say that this example somehow cheats,
I think the cheating part is that this category just uses matrix multiplication
to make every property necessary to form a category hold.

\subsubsection{With the appropriate notion of composition,
 graphs and their morphisms form a category}

Suppose we have three directed graphs, namely $(N,E), (M,F), (O,G)$:

\begin{diagram}
  E & \pile{\rTo^{\sigma_0} \\ \rTo_{\tau_0}} & N & \
  F & \pile{\rTo^{\sigma_1} \\ \rTo_{\tau_1}} & M & \
  G & \pile{\rTo^{\sigma_2} \\ \rTo_{\tau_2}} & O & \
\end{diagram}

and two morphisms $f,g$ between them:

\begin{diagram}
  (N,E) & \rTo^f & (M,F) & \rTo^g & (O,G)
\end{diagram}

Graph morphism $f$ consists of two functions $f_0, f_1$, such that:

\begin{align*}
  \sigma_1 \fcomp f_1 = f_0 \fcomp \sigma_0 \qquad \
  \tau_1 \fcomp f_1 = f_0 \fcomp \tau_0
\end{align*}

Similarly, for graph morphism $g$ we have:

\begin{align*}
  \sigma_2 \fcomp g_1 = g_0 \fcomp \sigma_1 \qquad \
  \tau_2 \fcomp f_1 = f_0 \fcomp \tau_1
\end{align*}

Suppose $E_0, F_0, G_0$ are edges in $E,F,G$ respectively
and $N_0 = \sigma_0(E_0), M_0 = \sigma_1(F_0), O_0 = \sigma_2(G_0)$.

We have the following diagram:

\begin{diagram}
  E_0             & \rTo^{f_1} & F_0             & \rTo^{g_1} & G_0 \\
  \dTo^{\sigma_0} &            & \dTo^{\sigma_1} &            & \dTo^{\sigma_2} \\
  N_0             & \rTo^{f_0} & M_0             & \rTo^{g_0} & O_0
\end{diagram}

Since $\sigma_1 \fcomp f_1 = f_0 \fcomp \sigma_0$
and $\sigma_2 \fcomp g_1 = g_0 \fcomp \sigma_1$,
we know that $g_0 \fcomp \sigma_1 \fcomp f_1 = g_0 \fcomp f_0 \fcomp \sigma_0 \
= \sigma_2 \fcomp g_1 \fcomp f_1$.

Similarly, we can conclude that $g_0 \fcomp f_0 \fcomp \tau_0 \
= \tau_2 \fcomp g_1 \fcomp f_1$.

If we denote graph morphisms as pairs of functions, then $f = (f_0,f_1)$ and $g = (g_0,g_1)$.
And we can define the composition of $f$ and $g$ as:

\begin{equation*}
  g \fcomp f = (g_0 \fcomp f_0, g_1 \fcomp f_1)
\end{equation*}

And the composition preserves the property that the following holds:

\begin{align*}
  \sigma_2 \fcomp (g_1 \fcomp f_1) & = (g_0 \fcomp f_0) \fcomp \sigma_0 \\
  \tau_2 \fcomp (g_1 \fcomp f_1) & = (g_0 \fcomp f_0) \fcomp \tau_0
\end{align*}

Therefore we can verify that by taking graphs as objects, graph morphisms as arrows,
and the graph morphism composition defined above as arrow composition, we can form a category.
The associativity of composition holds since:

\begin{align*}
  & (h \fcomp g) \fcomp f \\
  = & ((h_0 \fcomp g_0) \fcomp f_0, (h_1 \fcomp g_1) \fcomp f_1) \\
  = & (h_0 \fcomp (g_0 \fcomp f_0), h_1 \fcomp (g_1 \fcomp f_1)) \\
  = & h \fcomp (g \fcomp f)
\end{align*}

And each identity arrow is the corresponding identity function viewed as graph morphism.

\subsubsection{With the obvious composition this does form a category}

The objects and arrows are given here. So we only need to find a way to compose arrows,
verify its associativity, and find identity arrows.

Consider two new compatible arrows:

\begin{diagram}
  (A,R) & \rTo^{(f,\phi)} & (B,S) & \rTo^{(g,\psi)} & (C,T)
\end{diagram}

where $A,B,C$ are \cat{A}-objects and $R,S,T$ are \cat{S}-objects.

The arrow composition can be defined as:

\begin{align*}
  (g,\psi) \fcomp (f,\phi) = (g \fcomp f, \phi \fcomp \psi)
\end{align*}

We can verify that:

\begin{diagram}
  A & \rTo^{g \fcomp f} & C & R & \lTo^{\phi \fcomp \psi} & T
\end{diagram}

Therefore arrow composition produces another arrow between $(A,R)$ and $(C,T)$.

Given 3 arbitrary compatible arrows $(f,\phi), (g,\psi), (h,\eta)$,
we need to show that the arrow composition is associativity:

\begin{align*}
  & (h,\eta) \fcomp ((g,\psi) \fcomp (f,\phi)) \\
  = & (h,\eta) \fcomp (g \fcomp f, \phi \fcomp \psi) \\
  = & (h \fcomp (g \fcomp f), (\phi \fcomp \psi) \fcomp \eta) \\
  = & ((h \fcomp g) \fcomp f, \phi \fcomp (\psi \fcomp \eta)) \\
  = & (h \fcomp g, \psi \fcomp \eta) \fcomp (f, \phi) \\
  = & ((h,\eta) \fcomp (g,\psi)) \fcomp (f,\phi)
\end{align*}

And each identity arrow is the corresponding pair of identity functions
viewed as a pair of arrows.

\subsubsection{With the obvious composition this does form a category (SetR)}

The arrow composition can be defined in the following way:

For two arbitrary compatible arrows $(f,\phi)$ and $(g,\psi)$, where

\begin{diagram}
  (A,R) & \rTo^{(f,\phi)} & (B,S) & \rTo^{(g,\psi)} & (C,T)
\end{diagram}

The arrow composition is:

\begin{align*}
  (g,\psi) \fcomp (f,\phi) = (g \fcomp f, \phi \fcomp \psi)
\end{align*}

We need to prove that the arrow composition produces an arrow from $(A,R)$ to $(C,T)$,
which has the following property:

\begin{align*}
  \forall a \in A, t \in T, (g \fcomp f)(a (\phi \fcomp \psi)(t)) = \
  (g \fcomp f)(a)t
\end{align*}

\begin{proof}
  The following facts are known:

  \begin{align*}
    \forall a \in A, s \in S, & f(a\phi(s)) = f(a)s \\
    \forall b \in B, t \in T, & g(b\psi(t)) = g(b)t
  \end{align*}

  Since $\psi(t) \in S$, we know that
  $ f(a \phi(\psi(t))) = f(a (\phi \fcomp \psi)(t)) = f(a) \psi(t) $.

  Therefore $\forall a \in A, t \in T$,

  \begin{align*}
    & (g \fcomp f)(a (\phi \fcomp \psi)(t)) \\
    = & g(f(a (\phi \fcomp \psi)(t)) ) \\
    = & g(f(a) \psi(t) ) & f(a) \in B \\
    = & g(f(a))t = (g \fcomp f)(a)t
  \end{align*}

\end{proof}

The associativity of arrow composition holds in the same way that
the arrow composition in exercise $1.3.1$ holds, and each identity arrow
is the corresponding pair of identity functions viewed as an arrow.

Therefore, using the arrow composition decribed above
gives a category.

\end{document}