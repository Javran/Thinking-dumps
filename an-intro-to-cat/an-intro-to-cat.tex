\documentclass[11pt]{article}

\usepackage{amsmath}

\usepackage{fancyhdr}
\usepackage{amssymb}
\usepackage{amsthm}

\title{My Exercise Answers to \\
Introduction to Category Theory}
\author{Javran Cheng}

\begin{document}

\maketitle

\newcommand{\cat}[1]{\textbf{#1}}
\newcommand{\fcomp}{\circ}

\section{Categories}

\subsection{Categories defined}

\subsubsection{Observe that sets and functions do form a category \cat{Set}}

\begin{itemize}
  \item Objects are sets.
  \item Arrows are total functions, because function are just binary relations
    between two sets.
  \item Composition is the function composition.
  \item Associativity holds because $(h \fcomp g) \fcomp f = h \fcomp (g \fcomp f)$
    always holds for compatible functions.
  \item Identity arrows are functions that relates each element of a set to itself.
    Therefore the requirement $id_B \fcomp f = f = f \fcomp id_A$ becomes trivial.
\end{itemize}



\end{document}