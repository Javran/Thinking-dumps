\documentclass[11pt]{article}

\usepackage{amsmath}

\usepackage{fancyhdr}
\usepackage{amssymb}
\usepackage{amsthm}

\usepackage{diagrams}

\title{My Exercise Answers to \\
Introduction to Category Theory}
\author{Javran Cheng}

\begin{document}

\maketitle

\newcommand{\cat}[1]{\textit{#1}}
\newcommand{\fcomp}{\circ}

\newcommand{\nat}{\mathbb{N}}

\section{Categories}

\subsection{Categories defined}

\subsubsection{Observe that sets and functions do form a category \cat{Set}}

\begin{itemize}
  \item Objects are sets.
  \item Arrows are total functions, because function are just binary relations
    between two sets.
  \item Composition is the function composition.
  \item Associativity holds because $(h \fcomp g) \fcomp f = h \fcomp (g \fcomp f)$
    always holds for compatible functions.
  \item Identity arrows are functions that relates each element of a set to itself.
    Therefore the requirement $id_B \fcomp f = f = f \fcomp id_A$ becomes trivial.
\end{itemize}

\subsubsection{Each poset is a category, and each monoid is a category}

\paragraph{Each poset is a category} For any given poset, each element in it
is the corresponding object in the category. And each $\le$ relation corresponses
to an arrow in the category. Since $\le$ relation is transitive, the associativity
of arrow composition holds. And the identity for each object $A$ in the category is $(A,A)$,
this is because $A \le A$ holds (The binary relation is reflexive).

\paragraph{Each monoid is a category} For any given monoid $(M, \star , 1)$, the corresponding
category has only one element, which is the monoid itself. Every element in
this monoid is an arrow from the only object to itself. The arrow composition is
the binary relation $\star$. (We can observe that for any two arrows $f$ and $g$, $f \star g$
does form another arrow, and that $(f \star g) \star h = f \star (g \star h)$
is guaranteed by the property held by the monoid).

\subsection{Categories of structured sets}

\subsubsection{The category \cat{Pno}}

\paragraph{(a) Verify that \cat{Pno} is a category}

\begin{itemize}
  \item Objects are $(A, \alpha, a)$ described in the question.
  \item Arrows are functions $f : A \rTo B$ as described in the question
  \item Arrow composition.
    Suppose two arrows are:
    \begin{diagram}
      (A,\alpha,a) & & \rTo^f & &
      (B,\beta,b)  & & \rTo^g & &
      (C,\gamma,c)
    \end{diagram}
    Where $f$ and $g$ preserves the structure:
    \begin{align*}
      f \fcomp \alpha & = \beta  \fcomp f & \quad f(a) = b \\
      g \fcomp \beta  & = \gamma \fcomp g & \quad g(b) = c
    \end{align*}
    We can observe that:
    \begin{align*}
        & g \fcomp f \fcomp \alpha \\
      = & g \fcomp \beta  \fcomp f \\
      = & \gamma \fcomp g \fcomp f
    \end{align*}
    and that:
    \begin{align*}
      (g \fcomp f)(a) = g(b) = c
    \end{align*}

    Therefore the arrow composition is the function composition.

  \item Associativity holds by the function composition.
  \item For each object $(A,\alpha, a)$, $id_A \fcomp \alpha = \alpha \fcomp id_A$ and
    $id_A(a) = a$. So the identity arrow is the identity function on $A$ viewed as a morphism.
\end{itemize}

Therefore $\cat{Pno}$ is a category.

\paragraph{(b) Show that $(\nat, succ, 0)$ is a $\cat{Pno}$-object}

\begin{itemize}
  \item $\nat$ is a set.
  \item $succ$ is of type $\nat \rTo \nat$.
  \item $0 \in \nat$ is a nominated element.
\end{itemize}

Therefore, $(\nat, succ, 0)$ is a $\cat{Pno}$-object.

\paragraph{(c) Unique arrow and the behavior of the carrying function}\mbox{}

We know that $succ(0) = 1, succ(1) = 2, \ldots, succ(n) = n + 1$.
In order to describe the behavior of the carrying function $\alpha$,
We observe the value of $\alpha(f(0)), \alpha(f(1)), \ldots, \alpha(f(n))$:

\begin{align*}
\alpha(f(0)) & = (f \fcomp succ)(0) = (f \fcomp succ^1)(0) \\
\alpha(f(1)) & = \alpha(f \fcomp succ (0)) = (f \fcomp succ \fcomp succ)(0)
= (f \fcomp succ^2)(0) \\
& \ldots
\end{align*}

To sum up, $\forall n \in \nat$:
\begin{align*}
& \alpha(f(n)) \\
= &(\alpha \fcomp f)(succ^n(0)) \\
= &(\alpha \fcomp f \fcomp succ^n) (0) \\
= &(f \fcomp succ \fcomp succ^n) (0) \\
= &(f \fcomp succ^{n+1})(0) \\
= &f(n+1)
\end{align*}

And also:

\begin{align*}
& f(n) \\
= & (f \fcomp succ^{n})(0) \\
= & (\alpha^n \fcomp f)(0) \\
= & \alpha^n(a)
\end{align*}

Therefore the behavior of $\alpha$ is: $\forall a_1 \in A, \exists n_1 \in \nat$
s.t. $f(n_1) = a_1$, applying $\alpha$ to $a_1$ produces the same result as
applying $succ$ to $n_1$ and then applying $f$ to the resulting value does.

We can informally says that $\alpha$ to $(A,a)$ is like $succ$ to $(\nat,0)$.

The uniqueness of $f$ is also established, because from $f(n) = \alpha^n(a)$,
we know that $f$ is totally determined by $\alpha$ and $a$.

\subsubsection{Category $\cat{SetD}$}

\begin{itemize}
  \item Objects are of form $(A,X)$ which $A$ is a set and $X \subseteq A$.
  \item Arrows are functions of type $f: A \rTo B$ between two objects
    $(A,X)$ and $(B,Y)$
    where $\forall x \in X, f(x) \in Y$.
  \item Arrow composition. For any two compatible arrows $f$, $g$,
    we have the diagram:

    \begin{diagram}
      (A,X) & & \rTo^f & &
      (B,Y) & & \rTo^g & &
      (C,Z)
    \end{diagram}

    For arrow $f$ and $g$, the following constraints hold, respectively:

    \begin{align*}
      \forall x \in X, f(x) = y_x, y_x \in Y \quad \forall y \in Y, g(y) = z_y, z_y \in Z
    \end{align*}

    Therefore:

    \begin{align*}
      \forall x \in X, g(f(x)) = (g \fcomp f)(x) \in Z
    \end{align*}

    Therefore the arrow composition is function composition.

  \item Composition associativity. The associativity holds because
    the arrow composition is function composition.
  \item Identity arrow is the identity function viewed as a morphism.
\end{itemize}

\subsubsection{Show that these pairs are the objects of a category}

% TODO: notice the case that we mention category before we show that it is a category
%   need some modifications to eliminate this kind of mistakes

\begin{itemize}
  \item Objects are of form $(A,R)$, where $A$ is a set and $R \subseteq A \times A$
    is a binary relation on $A$.
  \item Arrows: Suppose two arbitrary objects in this category is $(A,R)$ and $(B,S)$.
    The arrow $f$:

    \begin{diagram}
      (A,R) & & \rTo^f & & (B,S)
    \end{diagram}

    is a function of type $f : A \times A \rTo B \times B$ such that
    $\forall a_1,a_2 \in A$, if $(a_1,a_2) \in R$, then $f(a_1,a_2) = (b_1,b_2) \in S$.
  \item Arrow composition: for any two compatible arrows $f$ and $g$:

    \begin{diagram}
      (A,R) & & \rTo^f & & (B,S) & & \rTo^g & & (C,T)
    \end{diagram}

    We know:

    \begin{align*}
      \forall (a_1,a_2) \in R, f(a_1,a_2) \in S \quad
      \forall (b_1,b_2) \in S, g(b_1,b_2) \in T
    \end{align*}

    Thus:

    \begin{align*}
      \forall (a_1,a_2) \in R, g(f(a_1,a_2)) = (g \fcomp f)(a_1,a_2) \in T
    \end{align*}

    Therefore the arrow composition is function composition.

  \item Composition associativity holds because the arrow composition
    is function composition.
  \item Identity arrow is the identity function viewed as a morphism.
\end{itemize}

\subsubsection{Category \cat{Top}}

Skip this exercise for a while.

\subsubsection{Show that $C[A,A]$ is a monoid under composition}

\begin{itemize}
  \item Elements in the set are all arrows in $C[A,A]$
  \item Binary operation $\star$ is arrow composition, which is function composition.
  \item The nominated element in the set is the identity arrow $id_A$.
  \item From the property of identity arrow, we know that $\forall s \in C[A,A]$,
    $id_A \star s = s = s \star id_A$.
  \item From the associativity enforced by the category,
    we know that $\forall s_1,s_2,s_3 \in C[A,A]$,
    $(s_1 \star s_2) \star s_3 = s_1 \star (s_2 \star s_3)$.

Therefore $C[A,A]$ is a monoid under composition.

\end{itemize}

\subsubsection{The composition of partial functions is associative}

Recall the diagram
for objects $A$, $B$, $C$ and arrows $f : A \rTo B$, $g : B \rTo C$:

\begin{diagram}
  A      & \rTo^f               & B       & \rTo^g               & C \\
  \uInto & \ruTo>{\bar{f}}      & \uInto  & \ruTo>{\bar{g}} \\
  X      &                      & Y       &              \\
  \uInto & \ruTo>{\bar{f}_{|U}} \\
  U
\end{diagram}

Now we add another object $D$ and an arrow $h : C \rTo D$,
and show a similar diagram about objects $B$, $C$, $D$ and arrows $g$, $h$:

\begin{diagram}
  B      & \rTo^g               & C       & \rTo^h               & D \\
  \uInto & \ruTo>{\bar{g}}      & \uInto  & \ruTo>{\bar{h}} \\
  Y      &                      & Z       &              \\
  \uInto & \ruTo>{\bar{g}_{|V}} \\
  V
\end{diagram}

Here $Z$ is the domain of total function $\bar{h}$, which is determined by $h$,
$V$ is a subset of $Y$, such that:

\begin{align*}
b \in V \Leftrightarrow b \in Y \text{ and } \bar{g}(b) \in Z
\end{align*}

Observe the two diagrams above we can find out the ``triangle'' between $B$, $Y$, $C$
overlaps, and the corresponding objects and arrows are equivalent.
So we can combine this two diagrams together to form a larger one:

\begin{diagram}
  A      & \rTo^f               & B       & \rTo^g               & C      & \rTo^h & D \\
  \uInto & \ruTo>{\bar{f}}      & \uInto  & \ruTo>{\bar{g}}      & \uInto & \ruTo>{\bar{h}} \\
  X      &                      & Y       &                      & Z \\
  \uInto & \ruTo>{\bar{f}_{|U}} & \uInto  & \ruTo>{\bar{g}_{|V}} \\
  U      &                      & V \\
  \uInto & \ruTo>{\bar{f}_{|W}} \\
  W
\end{diagram}

In the diagram above, $W$ is a subset of $U$, such that:

\begin{align*}
a \in W \Leftrightarrow a \in U \text{ and } \bar{f}(a) \in V
\end{align*}

Now we have 3 total functions $\bar{f}_{|W}$, $\bar{g}_{|V}$ and $\bar{h}$,
which are compatible for function composition. Thus we can conclude that:

\begin{align*}
\overline{(f \fcomp g) \fcomp h} =
(\bar{f}_{|W} \fcomp \bar{g}_{|V}) \fcomp \bar{h} =
\bar{f}_{|W} \fcomp (\bar{g}_{|V} \fcomp \bar{h}) =
\overline{f \fcomp (g \fcomp h)}
\end{align*}

Therefore for $\cat{Pfn}$, the arrow composition (the partial function composition)
is associative.

{%
\newcommand{\setbot}{\cat{$Set_\bot$}}
\subsubsection{Show that \setbot and \cat{Pfn}
are ``essentially the same'' category}

In order to see that two categories are ``essentially the same'',
we need to show that \setbot can be converted to \cat{Pfn}
and also that \cat{Pfn} can be converted to \setbot

In \setbot, the function $\phi : S \rTo T$ is a total function,
and we can restrict the domain of $\phi$ to ${\bot}$.
By adding this restriction, $\phi_\bot(x)$ is not defined for $a \in (S - \{\bot\})$
and thus $\phi_\bot$ is now a partial function. And keeping the same objects but
changing every arrrows from $f : S \rTo T$ to $f_\bot : \bot_S \rTo \bot_T$
will convert a \setbot into a \cat{Pfn}, which is proved to be a category.

On the other hand, for any two object $S$ and $T$ in \cat{Pfn}, we divide it into two parts:
$X$ and $(S - X)$, $Y$ and $(T - Y)$, respectively.
And for arrows we only take functions whose domains are $X$ and codomains are $Y$.
By taking these arrows, we can guaranteed the property that $\phi(\bot_S) = \bot_T$
by taking the set $X$ as $\bot_S$, $Y$ as $\bot_T$, a ``modified version'' of
the original arrow as $\phi$(See below).

By saying ``modified version'', I mean the original arrow is not a total function
from $S$ to $T$, however $\phi$ are required to be total. The property of pointed
set does not say anything about elements other than $\bot$. Therefore we can just
make some deterministic decisions to make elements in $S - X$ into $T$ to make
the arrow a total function. By doing this, we form a \setbot.

This is by no means a proper proof or even a reasonable observation,
because there are still some problems in my interpretation. For example,
you cannot convert \cat{Pfn} into \setbot and convert it back to get the same category.
And the other way around does not work neither. This is just the best attempt I have so far.
}%

{%
\newcommand{\rset}{R-\cat{Set}}
\newcommand{\setr}{\cat{Set}-R}
\subsubsection{Verify that \rset and \setr are categories}

Let $(R)$ be a fixed, but arbitrary monoid.

First we focus on \rset.

\begin{itemize}
  \item Objects are $R$-sets.
  \item Arrows are functions. Consider two objects
    and an arrow in this structured set:
    \begin{diagram}
      A & \rTo^f & B
    \end{diagram}
    For any element $a \in A$, let $f(a) = b$ where $b \in B$.
    Let $r,s \in R$. We need to verify that the following property holds:
    \begin{align*}
      s(rb) = (sr)b \quad
      1b = b
    \end{align*}
    Observe that:
    \begin{align*}
      s(rb) & = s(r f(a)) = s(f(ra)) = f(s(ra)) \\
      (sr)b & = (sr)f(a) = f((sr)a) = f(s(ra))
    \end{align*}
    In addition:
    \begin{align*}
      1b = 1f(a) = f(1a) = f(a) = b
    \end{align*}
    Therefore the function $f$ is a valid arrow between two objects.
  \item Arrow composition is function composition.
    Consider combining two arrows $f : A \rTo B$ and $g : B \rTo C$.
    Let $f(a) = b$, $g(b) = c$, where $a \in A, b \in B, c \in C$.
    Then we have the following properties:
    \begin{align*}
      f(ra) = rf(a) \quad g(rb) = rg(b)
    \end{align*}
    where $r \in R$.

    Consider the resulting arrow of composing $f$ and $g$:
    \begin{align*}
      (g \fcomp f)(ra) &= g(f(ra)) \\
      r (g \fcomp f)(a) & = r(g(f(a))) = g(r(f(a))) = g(f(ra))
    \end{align*}

    Therefore $g \fcomp f$ is a valid arrow that preserves the property.
  \item The associativity for arrows holds because of the associativity
    of function composition.
  \item The identity arrow is the identity function viewed as a morphism.
\end{itemize}

}%
\end{document}