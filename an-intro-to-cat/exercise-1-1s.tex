\subsection{Categories defined}

\subsubsection{Observe that sets and functions do form a category \cat{Set}}

\begin{itemize}
  \item Objects are sets.
  \item Arrows are total functions, because function are just binary relations
    between two sets.
  \item Composition is the function composition.
  \item Associativity holds because $(h \fcomp g) \fcomp f = h \fcomp (g \fcomp f)$
    always holds for compatible functions.
  \item Identity arrows are functions that relates each element of a set to itself.
    Therefore the requirement $id_B \fcomp f = f = f \fcomp id_A$ becomes trivial.
\end{itemize}

\subsubsection{Each poset is a category, and each monoid is a category}

\paragraph{Each poset is a category} For any given poset, each element in it
is the corresponding object in the category. And each $\le$ relation corresponses
to an arrow in the category. Since $\le$ relation is transitive, the associativity
of arrow composition holds. And the identity for each object $A$ in the structured set is $(A,A)$,
this is because $A \le A$ holds (The binary relation is reflexive).

\paragraph{Each monoid is a category} For any given monoid $(M, \star , 1)$, the corresponding
category has only one object (Here we don't need to care about
what exactly the object is, which isn't important). Every element in
this monoid is an arrow from the only object to itself.
The arrow composition is
the binary relation $\star$. We can observe that for any two arrows $f$ and $g$, $f \star g$
does form another arrow, and that $(f \star g) \star h = f \star (g \star h)$
is guaranteed by the property held by the monoid.
And because the unit $1$ correponses to the identity arrow,
the requirement that $1 \star m = m = m \star 1$ holds trivially (where $m$ are
elements in the monoid).