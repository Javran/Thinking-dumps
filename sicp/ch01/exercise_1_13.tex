\documentclass{article}
\usepackage{amsmath, amssymb, amsfonts, amsthm}
\begin{document}

\section*{Proof of exercise 1.13 in SICP}


When $n=0$
\begin{equation*}
	\frac{\phi^0-\psi^0}{\sqrt{5}} = 0 = Fib(0)
\end{equation*}

When $n=1$
\begin{equation*}
	\frac{\phi^1-\psi^1}{\sqrt{5}} = 1 = Fib(1)
\end{equation*}

Assume when $n=k$
\begin{equation*}
	Fib(k) = \frac{\phi^k-\psi^k}{\sqrt{5}}
\end{equation*}

\begin{align*}
	Fib(k-1) + Fib(k) 
	& = \frac{\phi^{k-1}-\psi^{k-1}}{\sqrt{5}} + \frac{\phi^k-\psi^k}{\sqrt{5}} \\
	& = \frac{1}{\sqrt{5}} \left ( \left ( \phi^{-1}+1 \right )\phi^k + \left ( \psi^{-1}+1 \right )\psi^k \right ) \\
	& = \frac{\phi^{k+1}-\psi^{k+1}}{\sqrt{5}} \\
	& = Fib(k+1)
\end{align*}

Then we need to prove:
\begin{equation*}
	\left | Fib(n)-\frac{\phi^n}{\sqrt{5}}\right | < \frac{1}{2}
\end{equation*}

\begin{align*}
	\left | Fib(n)-\frac{\phi^n}{\sqrt{5}} \right | & = \left | \frac{\psi^n}{\sqrt{5}} \right | \\
	& = \frac{ \left ( \frac{\sqrt{5}-1}{2} \right )^n }{\sqrt{5}} < \frac{1}{2}
\end{align*}

\end{document}
