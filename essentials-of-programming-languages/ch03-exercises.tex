\documentclass[11pt,a4paper]{article}
\usepackage{amsmath}
\usepackage{amsthm}
\usepackage{indentfirst}
\usepackage{bussproofs}

\DeclareMathOperator{\listofint}{List-of-int}

\title{Essentials of Programming Languages\\
{\large
My exercise answers}}
\author{Javran Cheng}
\begin{document}
\maketitle

\section{Exercise 3.1}

There are 4 places where the fact is used:

\begin{itemize}
    \item$  \lceil \dots \lfloor
                \texttt{(value-of <<x>> $\rho$)}
            \rfloor \dots \rceil
         \Rightarrow
            \lceil \dots
                \texttt{10}
            \dots \rceil$
    \item$  \lceil \dots \lfloor
                \texttt{(value-of <<3>> $\rho$)}
            \rfloor \dots \rceil
         \Rightarrow
            \lceil \dots
                \texttt{3}
            \dots \rceil$
    \item$  \lceil \dots \lfloor
                \texttt{(value-of <<v>> $\rho$)}
            \rfloor \dots \rceil
         \Rightarrow
            \lceil \dots
                \texttt{5}
            \dots \rceil$
    \item$  \lceil \dots \lfloor
                \texttt{(value-of <<i>> $\rho$)}
            \rfloor \dots \rceil
         \Rightarrow
            \lceil \dots
                \texttt{1}
            \dots \rceil$

\end{itemize}

\section{Exercise 3.2}

Not sure if this one is correct:

let $val = \texttt{(bool-val \#t)}$.

\begin{align*}
    \lfloor val \rfloor & = \bot \\
    \lceil \lfloor val \rfloor \rceil  & = \bot \\
    \texttt{(bool-val \#t)} & \neq \bot
\end{align*}
    
\section{Exercise 3.3}

Maybe it is because that substration is neither associative nor commutative.

So we can tell the difference between \texttt{-(3,2)} and \texttt{-(2,3)} by simply
looking at their results. It is more convenient to test for correctness.

The derivation is too long, so I would use $A$,$B$ and $C$ to shorten the line.

Where:

\begin{align*}
    A &: \texttt{(value-of <<-(x,11)>>) $= \lceil 22 \rceil $ } \\
    B &: \texttt{(value-of <<zero?(-(x,11))>>) $=$ \#f} \\
    C &: \texttt{(value-of <<if zero?(-(x,11)) then -(y,2) else -(y,4)>> $\rho$) }
\end{align*}

\begin{prooftree}
    \AxiomC{A}
    \AxiomC{
        $ \lceil 22 \rceil \neq \lceil 0 \rceil $
    }
    \BinaryInfC{B}
    \AxiomC{C}
    \BinaryInfC{
        \texttt{(value-of <<-(y,4)>> $\rho$) }
    }
    \UnaryInfC{
        $ \lceil 18 \rceil $
    }
\end{prooftree}


\end{document}
