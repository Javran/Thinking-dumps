\documentclass[11pt,a4paper]{article}
\usepackage{amsmath}
\usepackage{indentfirst}

\title{Essentials of Programming Languages\\
{\large
My exercise answers}}
\author{Javran Cheng}
\begin{document}
\maketitle

\section{Exercise 1.1}

\subsection{$\{ 3n + 2 | n \in N  \}$}

\subsubsection{top-down definition}

A natural number $n$ is in $S$ if and only if

\begin{itemize}
    \item{$n=2$, or}
    \item{$n-3 \in S$.}
\end{itemize}

\subsubsection{bottom-up definition}

Define the set $S$ to be the smallest set contained in $N$
and satisfying the following two properties:

\begin{itemize}
    \item{$2 \in S$, and}
    \item{if $n \in S$, then $n+3 \in S$.}
\end{itemize}

\subsubsection{rules of inference}

\begin{equation*}
    2 \in S
\end{equation*}

\begin{equation*}
    \frac{n \in S}{(n+3) \in S}
\end{equation*}

\subsection{$\{2n+3m+1|n,m\in N\}$}

\subsubsection{top-down definition}

A natural number $p$ is in $S$ if and only if

\begin{itemize}
    \item{$p=1$, or}
    \item{$p-2 \in S$, or}
    \item{$p-3 \in S$.}
\end{itemize}

\subsubsection{bottom-up definition}

Define the set $S$ to be the smallest set contained in $N$
and satisfying the following two properties:

\begin{itemize}
    \item{$1 \in S$, and}
    \item{if $p \in S$, then $p+2 \in S$ and $p+3 \in S$.}
\end{itemize}

\subsubsection{rules of inference}

\begin{equation*}
    1 \in S
\end{equation*}

\begin{equation*}
    \frac{n \in S}{(n+2) \in S}
\end{equation*}

\begin{equation*}
    \frac{n \in S}{(n+3) \in S}
\end{equation*}

\subsection{$\{(n,2n+1)| n \in N\}$}

\subsubsection{top-down definition}

A tuple of natural numbers $(n,m)$ is in $S$ if and only if

\begin{itemize}
    \item{$(n,m) = (0,1)$, or}
    \item{$(n-1,m-2) \in S $.}
\end{itemize}

\subsubsection{bottom-up definition}

Define the set $S$ to be the smallest set contained in $N^2$
and satisfying the following two properties:

\begin{itemize}
    \item{$(0,1) \in S$, and}
    \item{if $(n,m) \in S$, then $(n+1,m+2) \in S$.}
\end{itemize}

\subsubsection{rules of inference}

\begin{equation*}
    (0,1) \in S
\end{equation*}

\begin{equation*}
    \frac
        {(n,m) \in S}
        {(n+1,m+2) \in S}
\end{equation*}

\subsection{$\{(n,n^2)|n \in N\}$}

\subsubsection{top-down definition}

A tuple of natural numbers $(n,m)$ is in $S$ if and only if

\begin{itemize}
    \item{$(n,m) = (0,0)$, or}
    \item{$(n-1,m-2n+1) \in S $.}
\end{itemize}

\subsubsection{bottom-up definition}

Define the set $S$ to be the smallest set contained in $N^2$
and satisfying the following two properties:

\begin{itemize}
    \item{$(0,0) \in S$, and}
    \item{if $(n,m) \in S$, then $(n+1,m+2n+1) \in S$.}
\end{itemize}

\subsubsection{rules of inference}

\begin{equation*}
    (0,0) \in S
\end{equation*}

\begin{equation*}
    \frac
        {(n,m) \in S \quad n \in N \quad m \in N}
        {(n+1,m+2n+1) \in S}
\end{equation*}

\end{document}
