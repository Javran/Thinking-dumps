\documentclass[11pt,a4paper]{article}
\usepackage{amsmath}
\usepackage{amsthm}
\usepackage{indentfirst}

\DeclareMathOperator{\listofint}{List-of-int}

\title{Essentials of Programming Languages\\
{\large
My exercise answers}}
\author{Javran Cheng}
\begin{document}
\maketitle

\section{Exercise 1.1}

\subsection{$\{ 3n + 2 | n \in N  \}$}

\subsubsection{top-down definition}

A natural number $n$ is in $S$ if and only if

\begin{itemize}
    \item{$n=2$, or}
    \item{$n-3 \in S$.}
\end{itemize}

\subsubsection{bottom-up definition}

Define the set $S$ to be the smallest set contained in $N$
and satisfying the following two properties:

\begin{itemize}
    \item{$2 \in S$, and}
    \item{if $n \in S$, then $n+3 \in S$.}
\end{itemize}

\subsubsection{rules of inference}

\begin{equation*}
    2 \in S
\end{equation*}

\begin{equation*}
    \frac{n \in S}{(n+3) \in S}
\end{equation*}

\subsection{$\{2n+3m+1|n,m\in N\}$}

\subsubsection{top-down definition}

A natural number $p$ is in $S$ if and only if

\begin{itemize}
    \item{$p=1$, or}
    \item{$p-2 \in S$, or}
    \item{$p-3 \in S$.}
\end{itemize}

\subsubsection{bottom-up definition}

Define the set $S$ to be the smallest set contained in $N$
and satisfying the following two properties:

\begin{itemize}
    \item{$1 \in S$, and}
    \item{if $p \in S$, then $p+2 \in S$ and $p+3 \in S$.}
\end{itemize}

\subsubsection{rules of inference}

\begin{equation*}
    1 \in S
\end{equation*}

\begin{equation*}
    \frac{n \in S}{(n+2) \in S}
\end{equation*}

\begin{equation*}
    \frac{n \in S}{(n+3) \in S}
\end{equation*}

\subsection{$\{(n,2n+1)| n \in N\}$}

\subsubsection{top-down definition}

A tuple of natural numbers $(n,m)$ is in $S$ if and only if

\begin{itemize}
    \item{$(n,m) = (0,1)$, or}
    \item{$(n-1,m-2) \in S $.}
\end{itemize}

\subsubsection{bottom-up definition}

Define the set $S$ to be the smallest set contained in $N^2$
and satisfying the following two properties:

\begin{itemize}
    \item{$(0,1) \in S$, and}
    \item{if $(n,m) \in S$, then $(n+1,m+2) \in S$.}
\end{itemize}

\subsubsection{rules of inference}

\begin{equation*}
    (0,1) \in S
\end{equation*}

\begin{equation*}
    \frac
        {(n,m) \in S}
        {(n+1,m+2) \in S}
\end{equation*}

\subsection{$\{(n,n^2)|n \in N\}$}

\subsubsection{top-down definition}

A tuple of natural numbers $(n,m)$ is in $S$ if and only if

\begin{itemize}
    \item{$(n,m) = (0,0)$, or}
    \item{$(n-1,m-2n+1) \in S $.}
\end{itemize}

\subsubsection{bottom-up definition}

Define the set $S$ to be the smallest set contained in $N^2$
and satisfying the following two properties:

\begin{itemize}
    \item{$(0,0) \in S$, and}
    \item{if $(n,m) \in S$, then $(n+1,m+2n+1) \in S$.}
\end{itemize}

\subsubsection{rules of inference}

\begin{equation*}
    (0,0) \in S
\end{equation*}

\begin{equation*}
    \frac
        {(n,m) \in S \quad n \in N \quad m \in N}
        {(n+1,m+2n+1) \in S}
\end{equation*}

\section{Exercise 1.2}

\subsection{Q1}

As $n$ keeps increasing, we define function $f$:

\begin{equation*}
    f(n) =
    \begin{cases}
        (0,1) & n = 0, \\
        ((n-1)+1, k'+7) & n > 0 \text{ and } f(n-1) = (n-1,k').
    \end{cases}
\end{equation*}

Let $g(n) = cdr(f(n))$, thus:

\begin{align*}
    car(f(n)) & = n \\
    cdr(f(n)) & = g(n) = g(n-1) + 7 (n > 0)
\end{align*}

Therefore, the answer is:

\begin{equation*}
    \{ (n,7n+1) | n \in N \}
\end{equation*}

\subsection{Q2}

Define $f$ and $g$:

\begin{equation*}
    (f(n), g(n)) =
    \begin{cases}
        (0,1) & n = 0 \\
        ((n'+1, 2k') & n > 0 \text{ and } (f(n-1),g(n-1)) = (n',k')
    \end{cases}
\end{equation*}

For $n > 0$, we have:

\begin{align*}
    f(n) & = f(n-1) + 1 \\
    g(n) & = g(n-1) * 2 
\end{align*}

And we can further conclude that:

\begin{align*}
    f(n) & = n \\
    g(n) & = 2^n 
\end{align*}

Therefore, the answer is:

\begin{equation*}
    \{ (n,2^n) | n \in N \}
\end{equation*}

\subsection{Q3}

Define $f$, $g$ and $h$:

\begin{equation*}
    (f, g, h)(n) =
    \begin{cases}
        (0,0,1) & n = 0 \\
        ((n'+1, j', i'+j') & n > 0 \text{ and }
            (f,g,h)(n-1) = (n',i',j')
    \end{cases}
\end{equation*}

For $n > 0$, we have:

\begin{align*}
    f(n) & = f(n-1) + 1 \\
    g(n) & = h(n-1) \\
    h(n) & = g(n-1) + h(n-1) = h(n-2) + h(n-1)
\end{align*}

And we can further conclude that:

\begin{align*}
    f(n) & = n \\
    g(n) & = h(n-1) \\
    h(n) & = h(n-2) + h(n-1)
\end{align*}

Let $fib(n)$ denotes the $n^{th}$ Fibonacci number, $fib(0) = 0, fib(1) = 1$.

Therefore, the answer is:

\begin{equation*}
    \{ (n,fib(n),fib(n+1)) | n \in N \}
\end{equation*}

\subsection{Q4}

Define $f$, $g$ and $h$:

\begin{equation*}
    (f, g, h)(n) =
    \begin{cases}
        (0,1,0) & n = 0 \\
        ((n'+1, i'+2, i'+j') & n > 0 \text{ and }
            (f,g,h)(n-1) = (n',i',j')
    \end{cases}
\end{equation*}

For $n > 0$, we have:

\begin{align*}
    f(n) & = f(n-1) + 1 \\
    g(n) & = g(n-1) + 2 \\
    h(n) & = g(n-1) + h(n-1)
\end{align*}

And we can further conclude that:

\begin{align*}
    f(n) & = n \\
    g(n) & = 2n+1 \\
    h(n) & = \sum_{i=0}^{n-1} g(i) = 1 \cdot n + \frac{(n-1)n}{2} \cdot 2 = n^2
\end{align*}

Therefore, the answer is:

\begin{equation*}
    \{ (n,2n+1,n^2) | n \in N \}
\end{equation*}

\section{Exercise 1.3}

Initialize $T = S \cup \{1\}$, and follow the definition of 1.1.2.

Now that
$S  = \{ 3n | n \in N \}$,
$T = S \cup \{3n+1 | n \in N\}$,
and $T \supset S$,  $ T \neq S $.

\section{Exercise 1.4}

\begin{align*}
    & \listofint \\
    \Rightarrow & (Int \cdot \listofint) \\
    \Rightarrow & (Int \cdot (Int \cdot \listofint)) \\
    \Rightarrow & (Int \cdot (Int \cdot (Int \cdot \listofint))) \\
    \Rightarrow & ( -7 \cdot (Int \cdot (Int \cdot \listofint))) \\
    \Rightarrow & ( -7 \cdot (  3 \cdot (Int \cdot \listofint))) \\
    \Rightarrow & ( -7 \cdot (  3 \cdot ( 14 \cdot \listofint))) \\
    \Rightarrow & ( -7 \cdot (  3 \cdot ( 14 \cdot         ())))
\end{align*}

\section{Exercise 1.5}

If $e \in LcExp$, then there are the same number of left
and right parentheses in $e$.

\begin{proof}
    Let $f(x)$ be the number of steps required to formalize that $x \in LcExp$.
    We prove the theorem by induction on $f(e)$.

    Let $lp(x)$ and $rp(x)$ be the number of left parentheses and the number
    of right parentheses in $x$, respectively.

    The induction hypothesis $IH(k)$ is that
    any $x \in LcExp$ that $f(x) \le k$ has the same number of
    left parentheses and right parentheses.  By definition, $f(x) \ge 1$. 

    If $f(e) = 1$, then $e \in Identifier$.
    And $lp(e) = rp(e) = 0$, thus $IH(1)$ holds true.

    Let $IH(k)$ holds true, that is,
    for any $x \in LcExp$ that $f(x) \le k$,
    $lp(x) = rp(x)$ holds true.
    We need to show that $IH(k+1)$ holds as well,
    for all $e \in LcExp$ such that $f(e) = k+1$,
    we have three posibilities according to the definition of $LcExp$:

    \textbf{Case 1.} $e$ could be of form $Identifier$, which means only one
    step of formalization is required, $IH(k+1)$ holds for this case,
    especially, $k$ can only be $0$.

    \textbf{Case 2.} $e$ could be of form
    $(lambda \quad (Identifier) \quad LcExp)$. This struction has
    an inner $LcExp$, we name it $e'$. By definition of $f$,
    we know that $f(e') < f(e)$. And:

    \begin{align*}
        lp(e) & = 2 + lp(e') \\
        rp(e) & = 2 + rp(e')
    \end{align*}

    In addition, because $f(e') \le k$, we know $lp(e') = rp(e)$ by $IH(k)$.
    Therefore, in this case, $lp(e) = 2 + lp(e') = 2 + rp(e') = rp(e)$.
    $IH(k+1)$ holds.

    This completes the proof of the claim that $IH(k+1)$ holds and therefore
    completes the induction.

\end{proof}

\section{Exercise 1.6}

We will not have a chance to report the error.
And $car$ will complain about operating on an empty list.

\section{Exercise 1.7}

See ``./ex-1.7.rkt''

\section{Exercise 1.8}

See ``./ex-1.8.rkt''

\section{Exercise 1.9}

See ``./ex-1.9.rkt''

\section{Exercise 1.10}

``or'' can be exclusive, that is, $xor(A,B)$ holds if and only if
$or(A,B)$ holds but $A$ and $B$ cannot hold simultaneously. 

\section{Exercise 1.11}

\texttt{subst} is guaranteed to do recursion on a smaller substructure.
Because if \texttt{slist} is empty, it terminates, or it will call
\texttt{subst-in-s-exp} on \texttt{(car slist)} and call
\texttt{subst} on \texttt{(cdr slist)}, all of which are guaranteed to
be a smaller substructure and thus can halt.
\texttt{subst-in-s-exp} calls \texttt{subst} and thus is guaranteed to
terminate as well.

\section{Exercise 1.12}

See ``./ex-1.12.rkt''

\section{Exercise 1.13}

See ``./ex-1.13.rkt''

\section{Exercise 1.14}

\begin{proof}
We will prove this by induction.
The induction hypothesis $IH(n)$ is that
the result of \texttt{partial-vector-sum} is correct. i.e.:

\[
    \texttt{partial-vector-sum}(v,n) = \sum_{i=0}^{n}v_i
\]

We first show that $IH(0)$ holds. If $n=0$, the result is:

\[
    \texttt{(vector-ref v 0)} = v_0 = \sum_{i=0}^{0}v_i
\]

Therefore, $IH(0)$ holds.

Let $IH(k)$ holds true, we want to show that $IH(k+1)$ holds as well
to complete the induction.

Since $0 \le n < length(v)$, we know that $k + 1 \neq 0$, therefore:

\begin{align*}
    & \texttt{(+ (vector-ref v <k+1>) (partial-vector-sum v (- <k+1> 1)))} \\
    = & v_{k+1} + \texttt{(partial-vector-sum v k)} \\
    = & v_{k+1} + \sum_{i=0}^{k}v_i \\
    = & \sum_{i=0}^{k+1}v_i
\end{align*}

This completes the proof of the claim that $IH(k+1)$ holds and therefore
completes the induction.

\end{proof}

\end{document}
