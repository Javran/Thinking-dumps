\documentclass[11pt,a4paper]{article}
\usepackage{amsmath}
\usepackage{indentfirst}

\DeclareMathOperator{\listofint}{List-of-int}

\title{Essentials of Programming Languages\\
{\large
My exercise answers}}
\author{Javran Cheng}
\begin{document}
\maketitle

\section{Exercise 1.1}

\subsection{$\{ 3n + 2 | n \in N  \}$}

\subsubsection{top-down definition}

A natural number $n$ is in $S$ if and only if

\begin{itemize}
    \item{$n=2$, or}
    \item{$n-3 \in S$.}
\end{itemize}

\subsubsection{bottom-up definition}

Define the set $S$ to be the smallest set contained in $N$
and satisfying the following two properties:

\begin{itemize}
    \item{$2 \in S$, and}
    \item{if $n \in S$, then $n+3 \in S$.}
\end{itemize}

\subsubsection{rules of inference}

\begin{equation*}
    2 \in S
\end{equation*}

\begin{equation*}
    \frac{n \in S}{(n+3) \in S}
\end{equation*}

\subsection{$\{2n+3m+1|n,m\in N\}$}

\subsubsection{top-down definition}

A natural number $p$ is in $S$ if and only if

\begin{itemize}
    \item{$p=1$, or}
    \item{$p-2 \in S$, or}
    \item{$p-3 \in S$.}
\end{itemize}

\subsubsection{bottom-up definition}

Define the set $S$ to be the smallest set contained in $N$
and satisfying the following two properties:

\begin{itemize}
    \item{$1 \in S$, and}
    \item{if $p \in S$, then $p+2 \in S$ and $p+3 \in S$.}
\end{itemize}

\subsubsection{rules of inference}

\begin{equation*}
    1 \in S
\end{equation*}

\begin{equation*}
    \frac{n \in S}{(n+2) \in S}
\end{equation*}

\begin{equation*}
    \frac{n \in S}{(n+3) \in S}
\end{equation*}

\subsection{$\{(n,2n+1)| n \in N\}$}

\subsubsection{top-down definition}

A tuple of natural numbers $(n,m)$ is in $S$ if and only if

\begin{itemize}
    \item{$(n,m) = (0,1)$, or}
    \item{$(n-1,m-2) \in S $.}
\end{itemize}

\subsubsection{bottom-up definition}

Define the set $S$ to be the smallest set contained in $N^2$
and satisfying the following two properties:

\begin{itemize}
    \item{$(0,1) \in S$, and}
    \item{if $(n,m) \in S$, then $(n+1,m+2) \in S$.}
\end{itemize}

\subsubsection{rules of inference}

\begin{equation*}
    (0,1) \in S
\end{equation*}

\begin{equation*}
    \frac
        {(n,m) \in S}
        {(n+1,m+2) \in S}
\end{equation*}

\subsection{$\{(n,n^2)|n \in N\}$}

\subsubsection{top-down definition}

A tuple of natural numbers $(n,m)$ is in $S$ if and only if

\begin{itemize}
    \item{$(n,m) = (0,0)$, or}
    \item{$(n-1,m-2n+1) \in S $.}
\end{itemize}

\subsubsection{bottom-up definition}

Define the set $S$ to be the smallest set contained in $N^2$
and satisfying the following two properties:

\begin{itemize}
    \item{$(0,0) \in S$, and}
    \item{if $(n,m) \in S$, then $(n+1,m+2n+1) \in S$.}
\end{itemize}

\subsubsection{rules of inference}

\begin{equation*}
    (0,0) \in S
\end{equation*}

\begin{equation*}
    \frac
        {(n,m) \in S \quad n \in N \quad m \in N}
        {(n+1,m+2n+1) \in S}
\end{equation*}

\section{Exercise 1.2}

\subsection{Q1}

As $n$ keeps increasing, we define function $f$:

\begin{equation*}
    f(n) =
    \begin{cases}
        (0,1) & n = 0, \\
        ((n-1)+1, k'+7) & n > 0 \text{ and } f(n-1) = (n-1,k').
    \end{cases}
\end{equation*}

Let $g(n) = cdr(f(n))$, thus:

\begin{align*}
    car(f(n)) & = n \\
    cdr(f(n)) & = g(n) = g(n-1) + 7 (n > 0)
\end{align*}

Therefore, the answer is:

\begin{equation*}
    \{ (n,7n+1) | n \in N \}
\end{equation*}

\subsection{Q2}

Define $f$ and $g$:

\begin{equation*}
    (f(n), g(n)) =
    \begin{cases}
        (0,1) & n = 0 \\
        ((n'+1, 2k') & n > 0 \text{ and } (f(n-1),g(n-1)) = (n',k')
    \end{cases}
\end{equation*}

For $n > 0$, we have:

\begin{align*}
    f(n) & = f(n-1) + 1 \\
    g(n) & = g(n-1) * 2 
\end{align*}

And we can further conclude that:

\begin{align*}
    f(n) & = n \\
    g(n) & = 2^n 
\end{align*}

Therefore, the answer is:

\begin{equation*}
    \{ (n,2^n) | n \in N \}
\end{equation*}

\subsection{Q3}

Define $f$, $g$ and $h$:

\begin{equation*}
    (f, g, h)(n) =
    \begin{cases}
        (0,0,1) & n = 0 \\
        ((n'+1, j', i'+j') & n > 0 \text{ and }
            (f,g,h)(n-1) = (n',i',j')
    \end{cases}
\end{equation*}

For $n > 0$, we have:

\begin{align*}
    f(n) & = f(n-1) + 1 \\
    g(n) & = h(n-1) \\
    h(n) & = g(n-1) + h(n-1) = h(n-2) + h(n-1)
\end{align*}

And we can further conclude that:

\begin{align*}
    f(n) & = n \\
    g(n) & = h(n-1) \\
    h(n) & = h(n-2) + h(n-1)
\end{align*}

Let $fib(n)$ denotes the $n^{th}$ Fibonacci number, $fib(0) = 0, fib(1) = 1$.

Therefore, the answer is:

\begin{equation*}
    \{ (n,fib(n),fib(n+1)) | n \in N \}
\end{equation*}

\subsection{Q4}

Define $f$, $g$ and $h$:

\begin{equation*}
    (f, g, h)(n) =
    \begin{cases}
        (0,1,0) & n = 0 \\
        ((n'+1, i'+2, i'+j') & n > 0 \text{ and }
            (f,g,h)(n-1) = (n',i',j')
    \end{cases}
\end{equation*}

For $n > 0$, we have:

\begin{align*}
    f(n) & = f(n-1) + 1 \\
    g(n) & = g(n-1) + 2 \\
    h(n) & = g(n-1) + h(n-1)
\end{align*}

And we can further conclude that:

\begin{align*}
    f(n) & = n \\
    g(n) & = 2n+1 \\
    h(n) & = \sum_{i=0}^{n-1} g(i) = 1 \cdot n + \frac{(n-1)n}{2} \cdot 2 = n^2
\end{align*}

Therefore, the answer is:

\begin{equation*}
    \{ (n,2n+1,n^2) | n \in N \}
\end{equation*}

\section{Exercise 1.3}

Initialize $T = S \cup \{1\}$, and follow the definition of 1.1.2.

Now that
$S  = \{ 3n | n \in N \}$,
$T = S \cup \{3n+1 | n \in N\}$,
and $T \supset S$,  $ T \neq S $.

\section{Exercise 1.4}

\begin{align*}
    & \listofint \\
    \Rightarrow & (Int \cdot \listofint) \\
    \Rightarrow & (Int \cdot (Int \cdot \listofint)) \\
    \Rightarrow & (Int \cdot (Int \cdot (Int \cdot \listofint))) \\
    \Rightarrow & ( -7 \cdot (Int \cdot (Int \cdot \listofint))) \\
    \Rightarrow & ( -7 \cdot (  3 \cdot (Int \cdot \listofint))) \\
    \Rightarrow & ( -7 \cdot (  3 \cdot ( 14 \cdot \listofint))) \\
    \Rightarrow & ( -7 \cdot (  3 \cdot ( 14 \cdot         ())))
\end{align*}


\end{document}
